Solving partial differential equations (PDEs) is crucial for modeling various phenomena in physics, engineering, and beyond.
While approaches like traditional finite difference schemes provide accurate solutions, physics-informed neural networks

Specific blablabla

What have we done?

We have used the finite difference method and Physics-informed Neural Networks (PINNs) to numerically solve the one-dimensional heat equation. 
We find that the finite difference method outperforms the neural network. 
Its MSE compared to the analytical solution is $2.52 \times 10^{-7}$, whereas the neural network achieves $ 6.71 \times 10^{-6}$. 

Implications and further improvements

\mia{m}\julie{o}\gaute{r}\mia{e}