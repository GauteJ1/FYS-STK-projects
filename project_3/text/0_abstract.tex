Solving partial differential equations (PDEs) is crucial for modeling various phenomena in physics, engineering, and beyond.
While approaches like traditional finite difference schemes provide accurate solutions, physics-informed neural networks (PINNs) offer a promising alternative with greater flexibility for complex and high-dimensional PDEs.
This paper will compare the performance of a finite difference scheme and PINNs for solving the heat equation in one spatial dimension.
This is a much-used PDE within physics and provides a good basis for comparing different numerical solving methods for PDEs.
We implement a forward Euler finite difference method from scratch and use PyTorch's functionality to implement a physics-informed neural network.
We find that the finite difference method outperforms the neural network. 
Its MSE compared to the analytical solution is $2.52 \times 10^{-7}$, whereas the neural network achieves $ 6.71 \times 10^{-6}$. 
We conclude that the finite difference method is the preferred method for solving the heat equation, due to a slightly lower MSE and substantially lower computational cost.
However, we recognize that the PINN has greater potential in some practical and more advanced situations, and is easier to adapt to other PDEs when implemented.
Further research on the topic would require both testing the methods on more advanced PDEs, but also a deeper discussion on what traits should be valued in a PDE-solving method, as PINNs and finite difference methods each excel at different areas.