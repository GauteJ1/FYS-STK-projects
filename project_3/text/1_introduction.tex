In an increasingly data-driven world, precise models are of paramount importance.
As the demand for models of complex data has increased, the need for stable, efficient, and accurate modeling of real-world data is at an all-time high - and the world is constantly seeking new and better methods.

Partial differential equations (PDEs) provide a way to mathematically describe the relation between the values and rate of change in complex systems.
While PDEs may be most commonly associated with physics and mechanics, where they are derived from theory, they're also prevalent within other fields of application.
Both within macro-economics \cite{ecoPDE} and epidemiology, such as modeling the spread of COVID-19 \cite{covidus}, PDEs play an important role in describing the world around us. 

Despite a recent surge in popularity, PDEs have been a longstanding part of applied mathematics, with sources dating as far back as the 18th century \cite{Cajori}. For many of the most popular PDEs there exists analytical approaches proven to solve the equations. However, for more complex variants it's close to impossible to solve the equations by hand and we rely on numerical approximations to provide an answer. Particularly the introduction of finite difference schemes in the 20th century has proved to be of great importance \cite{sloan2012}. With the increased possibilities of computational tools, the importance of numerical approaches became concurrently evident. Within the past couple of years, PDEs have gained significant traction within the world of machine learning, especially in the field of physics-informed neural networks (PINNs).  
% Numerical methods have become an integral part of solving large-scale applications 

In this report, we start by presenting PDEs (Sec. \ref{sec:PDE}) and more specifically, the heat equation in Sec. \ref{sec:diffeq}.
We furthermore elaborate on the finite difference schemes and neural networks in Secs. \ref{sec:FD}, \ref{sec:NN} and \ref{sec:gd}. 
The last part of the theory section introduces a special variant of neural networks called PINNs (Sec. \ref{sec:PINNs}).
We continue by presenting our implementation of the finite difference method and PINNs for solving the one-dimensional heat equation for a rod of length $1$ in the methods section (Sec. \ref{sec:methods3}).
Our results from the numerical methods are compared to the analytical solution in Sec. \ref{sec:results_and_discussion}, where we also discuss aspects of the different methods and their relevant hyperparameters. 
Finally, our findings are concluded in Sec. \ref{sec:conclucion5}.

