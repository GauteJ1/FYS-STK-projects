In an increasingly data-driven world, precise models are of paramount importance.
As the demand for models of complex data has increased, the need for stable, efficient and accurate modeling of real world data is at an all time high - and the world is constantly seeking new and better methods.

Partial differential equations (PDEs) provide a way to mathematically describe the relation between the values and rate of change in complex systems.
While PDEs may be most known within physics and mechanics, where they are derived from theory, they are also prevalent within other applications.
Both within macro-economics \cite{ecoPDE} and epidemiology, such as modeling the spread of COVID-19 \cite{covidus}, PDEs play an important role in describing the world around us. 

PDEs has been a longstanding part of applied mathematics, with sources dating as far back as the ??. 

FD 1928 

\gaute{analytical vs numeric solutions}

\gaute{types of numerical solutions: NN vs finite difference}

\gaute{Some history stuff?}

\gaute{what we have done?}

