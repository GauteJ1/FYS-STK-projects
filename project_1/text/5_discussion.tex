\mia{move from result section down here}

Both the Franke function and the terrain data are chosen to have $41^2 = 1681$ data points. \mia{the impact of this}

By including only a small section of the terrain data, the area is 

\mia{discussion of whether it is even correct to compare the error measures (i.e. that the BS error measure is wrong) and that more data is utilized for the cv than for the bootstrap}

For the model trained with cross-validation, more data is utilized than for the one trained with bootstrap. In the ladder we split into a training and a test data set, whereas cross-validation trains on the entire dataset. \mia{continue ones the plot is here}

\mia{suggestions for further development: could implement bootstrap "corectly" (use different word) to have more data, non-linear methods}

Further work to do on this could include a different implementation of the bootstrap method. We could use the entire data set to draw bootstrap samples from. Then we would have to keep track of which samples where not included in the bootstrap sample and use these as the test set for the specific bootstrap model. This would allow us to use more data for training and improve the model. 

%Furthermore, it is interesting to discuss whether linear models are sufficient for these kinds of problems or not. \mia{continue}
Furthermore, non-linear methods could be utilized. 